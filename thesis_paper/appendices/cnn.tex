\chapter{Convolutional Neural Networks}
\label{appx:simulation}

Convolutional Neural Networks for image classification \cite{krizhevsky2012imagenet} take an image as an input, process it, and output category to which that image belongs. The processing part consists of a series of layers through which image is propagated in order to learn features, which in turn determine to which class an image belongs. (\textcolor{red}{\autoref{fig:cnn1}}).

\begin{figure}[h]
	\centering
	\includegraphics[scale=2]{cnn.jpg}
	\caption{Convolutional Neural Network consisting of two Convolutional layers, two Max-Pooling layers, Flatten layer and two Fully-Connected (Dense) layers, source \cite{alom2019state}}
	\label{fig:cnn1}
\end{figure}

The most commonly used layers in CNN architectures are convolutional layer, max-pooling layer, flatten layer, dense layer, and dropout layer.

\section{Convolutional Layer}

The convolutional layer is the building block of the CNN architecture. Its primary purpose is to extract features from an input image, such as edges, lines, curves, colors. As we go deeper inside the network, it starts identifying more complex features, such as shapes, objects. This layer consists of multiple filters (feature extractors, usually 3$\times$3 matrices) whose parameters need to be learned.
\section{Max-Pooling Layer}

The max-pooling layer is located after a series of convolutional layers in CNN architecture. It is a downsampling method that reduces dimensionality, thus decreasing the number of parameters and computational power needed in order to train the network, while retaining important features and patterns. It is achieved by applying a max filter to non-overlapping subregions (usually 2$\times$2 matrices), thus reducing the size of each feature map by a factor of 2.

\section{Flatten Layer}

The output of the convolutional base of the network (series of convolutional and max-pooling layers) is a two-dimensional matrix, and before feeding that data to the classification top of the network, it needs to be transformed. Flatten layer reshapes the output matrix to a vector, thus removing all dimensions but one in the process, making the data prepared for the series of fully-connected layers.

\section{Fully-Connected Layer}

After the high-level features of the image have been detected, a series of fully-connected (dense) layers are attached to the top of the network in order to classify an image into a label. Dense layers consist of a huge number of nodes (neurons), which provide a way of learning non-linear combinations of features outputted by a convolutional base, and determine which features most correlate to a particular class.

\section{Dropout Layer}

The fully-connected layer contains the most parameters in the network, and as a result neurons develop co-dependency amongst each other during training, which leads to overfitting the data (not generalizing well on new, unseen images). In order to prevent that, dropout layers are positioned right after dense layers in CNN architecture as a means of regularizing the network. Dropout consists of randomly ignoring (dropping out) fraction of neurons of fully-connected layer, which in turn makes network learn more robust features, and achieve better performance.
\clearpage
