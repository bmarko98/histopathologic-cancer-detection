\chapter{Conclusion}
\label{ch:sum}

Over the past decade convolutional neural networks have achieved results few expected would be possible in such a short time period, and recently they have been successfully implemented in healthcare industry. Histopathological Cancer Detection program shows that deep learning algorithms can be applied in the field of histopathology, and achieve high performance results in order reduce workload from doctors, and help speed up the process. More importantly, it overcomes one of the major drawbacks of deep learning algorithms, which is the notion of algorithms being 'black-boxes', i.e. not being highly interpretable. By implementing multiple techniques to further analyze network and its decisions, it is possible to explain and clarify network outputs and determine whether such reasoning is correct or not.

\section{Future Work}

After completing the thesis and creating program for histopathologic cancer detection on breast and colorectal tissue, there are multiple ways of extending the scope of the program: adding more tissue types in order to increase solving power, adding additional ways to further analyze results, improving graphical user interface (in order to make it even more simpler and easier to use).

ACDC@LUNGHP challenge provides dataset for lung cancer histopathologic detection, which could be used for extending the program scope. DigestPath2019 challenge provides dataset for colorectal histopathologic cancer detection, which could be used to improve accuracy of already existing CNNSimple network. Additional datasets are publicly available, such as ECDP2020 challenge dataset, ANHIR challenge dataset, etc.

Deconvolutional neural networks (DCNNs) could be used to visually interpret CNNs, which would help further analyze CNN classification results and increase interpretability of the network.
